\documentclass[a4paper]{article}
\usepackage[T1]{fontenc}
\usepackage[utf8]{inputenc}
\usepackage{lmodern}
\usepackage[francais]{babel}
\usepackage{listings}
\usepackage{color}
\usepackage{amsmath}

\definecolor{dkgreen}{rgb}{0,0.6,0}
\definecolor{gray}{rgb}{0.5,0.5,0.5}
\definecolor{mauve}{rgb}{0.58,0,0.82}

\title {Rapport Projet : SOKOBAN}
\author{Julien JACQUET 21400579}

\begin{document}
  \pagenumbering{gobble}
  \maketitle
  \newpage
  \pagenumbering{arabic}

  \paragraph{Note :}
  L'intégralité des fichiers nécéssaires a la compilation de ce pdf se trouvent dans le dossier Rapport.
  \paragraph{Note :}
  La partie creation de niveau est non fonctionnelle.
  \paragraph{Note :}
  En cas de tests avec valgrind il faut entrer:   \hfill \break
  valgrind make test   \hfill \break
  Sinon erreur : valgrind: mmap(0x315000, 1600327680) failed in UME with error 22 (Invalid argument).  \hfill \break
  (cependant cette methode entraine une perte de 10 octets a cause de la fonction make)   \hfill \break
  L'erreur semble venir de valgrind.

  \section{Explications des structures de données :}

  \lstset{
    language=C,               % choose the language of the code
    aboveskip=3mm,
    belowskip=3mm,
    basicstyle={\small\ttfamily},
    numbers=left,                   % where to put the line-numbers
    stepnumber=1,                   % the step between two line-numbers.
    numbersep=5pt,                  % how far the line-numbers are from the code
    backgroundcolor=\color{white},  % choose the background color. You must add \usepackage{color}
    showspaces=false,               % show spaces adding particular underscores
    showstringspaces=false,         % underline spaces within strings
    showtabs=false,                 % show tabs within strings adding particular underscores
    tabsize=2,                      % sets default tabsize to 2 spaces
    numberstyle=\tiny\color{gray},
    keywordstyle=\color{blue},
    commentstyle=\color{dkgreen},
    stringstyle=\color{mauve},
    captionpos=b,                   % sets the caption-position to bottom
    breaklines=true,                % sets automatic line breaking
    breakatwhitespace=true,         % sets if automatic breaks should only happen at whitespace
    %title=\lstname,                 % show the filename of files included with \lstinputlisting;
  }
  \lstinputlisting{structure_sokoban.c}
  \paragraph{}
  La structure $sokoban$ est composée d'un entier mode qui sera uniquement utilisé pour l'affichage.
  La partie importante de cette structure est la $CASE$, c'est un double tableau de 100 par 100, la forme de tableau semble convenir puisqu'il est aisé de représenter mentalement le sokoban sous forme matricielle.
  En dernier element de cette structure on a un entier niveau qui sert a retenir le niveau auquel on se trouve.
  La structure $une\_case$ n'est pas vraiment nécéssaire mais elle permet de clarifier la construction de la structure $sokoban$.
  Une chose a noter cependant :
  La strucuture $sokoban$ ne stockera jamais autre chose que des entiers, alors que le fichier $sasquatch1.xsb$ contient des $char$.
  La raison a cela est une grande simplification de la lecture du fichier, la lecture et le stockage des $char$ pouvant s'averer laborieux, uniquement le code ascii des caracteres est retenu.

  \lstinputlisting{structure_action.c}
  \paragraph{}
  La structure $action$ contient deux entiers un entier $mode$ qui définira les actions a effectuer (UNDO,REDO,QUIT,SUIVANT,etc) et un entier $fleche$ qui sert a renvoyer la valeur de la flèche quand l'utilisateur appuie dessus.

  La structure $coord$ contient deux entiers qui sont en fait les coodonnées de l'emplacement du joueur dans le $sokoban$.
  \lstinputlisting{structure_level.c}
  \paragraph{}
  La structure level a deux usages, le premier est de determiner la plus grande largeur et hauteur possible d'un niveau contenu dans un fichier. Ces donees servent dans la fonction $preprocess()$ et servent a determiner la taille de la fenetre graphique et les données qui en dépendent.
  Le second usage est de retenir la ligne de debut d'un niveau ainsi que la ligne de fin.

  \lstinputlisting{structure_stack.c}
  \paragraph{}
  Les structures $stackRedo$ et $stackUndo$ sont identiques, ce sont de piles, elles contiennent un tableau de type $sokoban$ pour stocker l'état du sokoban, et un int $top$ qui est necessaire pour créer une pile.

  \lstinputlisting{structure_creation.c}
  \paragraph{}
  Bien que le module de creation ne soit pas fonctionnel, la strucutre de creation de niveau existe, elle contient un $sokoban$ et un mode qui devait correspondre au type de caractere(mur , caisse), choisi par l'utilisateur, à ajouter dans le fichier.

  \section{Critiques :}
  \paragraph{}
  Certaines fonctions ont demandées un travail extremement chronophage, principalement la lecture prenant plusieurs jours pour ecrire une 150 lignes alors que d'autres (la partie des mouvements d'$action$), après 20 minutes de reflexion ont pris seulement une journée a écrire (400 lignes).
  Il est difficile d'évaluer a l'avance la difficultée que poseront certains modules du projet.
  D'autre part certaines fonctions du projet relevent du bricolage mais elles fonctionnent.
  Le module de creation ne fonctionne pas, quelques jours supplémentaires auraient étés nécéssaires malgré un mise au travail debut décembre.

  \section{Références :}
  \paragraph{}
  Pour la lecture : http://stackoverflow.com/questions/4179671/read-in-text-file-1-character-at-a-time-using-c  \hfill \break
  Pour les piles : l'intégralitée des resultats de la premiere page google de la recherche "stack in C". \hfill \break
  Pour les converions ascii : https://www.branah.com/ascii-converter

\end{document}
\grid
\grid
